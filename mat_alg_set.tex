\documentclass[main.tex]{subfiles}
\begin{document}
\begin{definition}[集合]
集合是若干个事物组成的整体。构成集合的事物或对象称为元素。集合还必须满足:
\begin{itemize}
    \item 无序性:一个集合中,每个元素的地位是相同的,元素之间是无序的
    \item 互异性:一个集合中,任何两个元素都不相同,即每个元素只出现一次
    \item 确定性:给定一个集合,某一事物要么属于该集合,要么不属于该集合,不能模棱两可。
\end{itemize}
\end{definition}

例:集合的表示方法。\(A=\{2,4,6,8\}\)。\(B=\{n|n\text{是偶数且}1<n<9\}\)。\(1\notin A\)。\(2\in B\)。

我们能否说上例的两个集合是“相同”的或“相等”的?下一条定义将明确这一问题的答案。

\begin{definition}[子集、包含]
如果\(a\in A\forall a\in B\),则称$B$是$A$的子集,或称$A$包含$B$,记为\(B\subseteq A\),\(A\supseteq B\)。如果\(A\subseteq B\)且\(B\subseteq A\)则称\(A=B\)。
\end{definition}

例:可验证

\begin{enumerate}
\item \(S\subseteq S \forall S\);
\item 若\(A\subseteq B\)且\(B\subseteq A\)则称\(A=b\)(可视为“两集合相等”的定义。)
\item 若\(A\subseteq B\)且\(B\subseteq C\)则\(A\subseteq C\)。
\end{enumerate}


\begin{definition}[真子集、真包含]
若\(B\subseteq A\)且\(B\neq A\),则称$B$是$A$的真子集,或称$A$真包含$B$,记为\(B\subset A\),\(A\supset B\)。
\end{definition}

例:可验证

\begin{enumerate}
    \item $S\subset S$不成立$\forall S$。
    \item 若$A\subset B$则$B\subset A$不成立。
    \item 若$A\subset B$且$B\subset C$,则$A\subset C$。
\end{enumerate}


\begin{definition}[空集]
空集$\emptyset = \{a|a\neq a\}$
\end{definition}

例:可验证

\begin{enumerate}
    \item $\emptyset\subseteq A,\forall A$。
    \item $\emptyset\not\subseteq A,\forall A\neq\emptyset$。
\end{enumerate}


\begin{definition}[并集]
若$a\in A$或$a\in B,\forall a,A,B$,则$a$属于$A$与$B$的并集,记为$a\in A\cup B$。
\end{definition}

例:可验证

\begin{enumerate}
    \item 结合律:$A\cup\left(B\cup C\right)=\left(A\cup B\right)\cup C$。
    \item 交换律:$A\cup B=B\cup A$。
    \item $\emptyset\cup A=A,\forall A$。
\end{enumerate}


\begin{definition}[交集]
若$a\in A$且$a\in B,\forall a,A,B$,则$a\in A\cap B$,称$A$与$B$的交集。
\end{definition}

\begin{definition}[笛卡尔积]
两个集合$A$和$B$的笛卡尔积是所有有序对$\left(a,b\right),a\in A,b\in B$的集合:

$A\times B=\{\left(a,b\right)|a\in A\text{且}b\in B\}$
\end{definition}

\begin{figure}[htbp]
\centering
\includegraphics[width=0.25\textwidth]{images/fig. 4.2.eps}
\caption{映射}
\label{fig:4.2}
\end{figure}

\begin{definition}[映射]
(如图\ref{fig:4.2}所示)从集合$X$到集合$Y$的映射$f$,记为$f:X\rightarrow Y$,是$X$的所有元素与$Y$的部分或所有元素之间的对应关系,且每个$X$的元素只对应$Y$的一个元素。如果$y\in Y$是$x\in X$通过映射$f$的对应,则可写成:
\[y=f\left(x\right)\text{或}x\mapsto f\left(x\right)\text{或}f:x\mapsto y\]
并称$f\left(x\right)$是$x$的像。按上述定义,$x$只有一个像。$X$称为定义域,$Y$称为陪域或到达域。$x$的像$f\left(x\right)$是$Y$的子集,叫值域。
\end{definition}

\begin{definition}[映射的相等]
若映射$f:X\rightarrow Y$和$g:X\rightarrow Y$满足$f\left(x\right)=g\left(x\right),\forall x\in X$,则这两个映射相等。
\end{definition}

\begin{definition}[函数]
陪域是$\mathbb{R}$或$\mathbb{C}$的映射称为函数。
\end{definition}

\begin{definition}[恒等映射]
恒等映射$\mathrm{id}_A:A\rightarrow A$是由集合$A$到其自身的如下映射:$\mathrm{id}_A\left(a\right)=a,\forall a\in A$。
\end{definition}

\begin{figure}[htbp]
\centering
\includegraphics[width=0.5\textwidth]{images/fig. 4.2a.eps}
\caption{满射、单射和双射}
\label{fig:sur_in_bijective}
\end{figure}

\begin{definition}[单射、双射、满射]
(如图\ref{fig:sur_in_bijective}所示)对于映射$f:X\rightarrow Y$,若$f\left(x_1\right)=f\left(x_2\right)\Rightarrow x_1=x_2$,则映射$f$是单射。若$f\left(X\right)=Y$,则称$f$是满射。既是单射又是满射的映射叫双射。
\end{definition}

\begin{definition}
令$f:X\rightarrow Y$和$g:Y\rightarrow Z$是两个映射。则$f$和$g$的复合映射,记为$g\circ f$,是从$X$到$Z$的映射:
\[g\circ f\left(x\right)=g\left(f\left(x\right)\right),\forall x\in X\]
\end{definition}

\begin{theorem}
对于映射$f:X\rightarrow Y$和$g: Y\rightarrow Z$,
\begin{itemize}
    \item 如果$f$和$g$都是满射,则$g\circ f$是满射
    \item 如果$g\circ f$是满射,则$g$是满射
\end{itemize}
\end{theorem}

\begin{theorem}
对于映射$f:X\rightarrow Y$和$g: Y\rightarrow Z$,
\begin{itemize}
    \item 如果$f$和$g$都是单射,则$g\circ f$是单射
    \item 如果$g\circ f$是满射,则$f$是单射
\end{itemize}
\end{theorem}

\begin{theorem}
双射必存在逆映射。双射的逆映射也是双射。
\end{theorem}
\end{document}