\documentclass[main.tex]{subfiles}
\begin{document}
第一部分有两个目的.一是为编程作些数学铺垫,是漫谈和热身,算不上是在严格意义上的数学准备.
二是通过一些实际例子说明,在与时俱进的今天,编程即使
在纯数学学习中也不是可有可无.在这些实际例子中,编程在数学中扮演着的不可缺少的角色.
第\ref{ch1}章想说明,美需要精确的数学表达.
第\ref{ch2}章则通过一些公式的图视证明,说明数学也可以更优美更简洁地表达出来.
但所涉及的计算,手工进行的话太繁重,而用编程则是轻而易举的.
这两章中的数据生成和图表绘制是以后各章的编程例子.
但愿达成一个共识,编程与作图作为一个工具,会成为数学想象的翅膀.

读者可把这两章先当着趣味数学阅读,在以后有关章节中实际编程时,再回到本章有关问题.
但是,动口不动手在教学上有明显的缺点.
所以,建议教师在实际讲课时,先从第\ref{partII}部分 Python 编程开始,
在编程和作图中再结合例子介绍第\ref{partI}部分的内容.
而学生灵活机动地预习这里的第\ref{partI}部分.



\end{document} 
