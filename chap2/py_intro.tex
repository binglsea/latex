\documentclass[main.tex]{subfiles}
\begin{document}

%Как говорят русские: Не ошибается тот, кто ничего не делает

不犯错者一事无成\footnote{{\fontencoding{OT2}\selectfont Ne oxibaet{}s{ya} tot, kto niqego ne lelaet.}}
俄国彦语

Python 语言的最初设计者和主要架构师吉多·范罗苏姆 (Guido van Rossum), 荷兰人,1982 年在阿姆斯特丹大学获得数学和计算机科学硕士学位。
据说 1989 年的圣诞节期间,为了打发时间而开发一个新的脚本\index{脚本, Script}解释器\index{解释器, Interpreter}(Script Interpreter),以替代使用Unix shell和C语言进行系统管理。
后来 Python 成了一种广泛使用的解释型、高级和通用的编程语言,支持多种编程范型,包括函数式、指令式、结构化、面向对象和反射式编程。
它拥有动态类型系统和垃圾回收功能,能够自动管理内存使用,并且其本身拥有一个巨大而广泛的标准库。
本节提到的名词术语,如果初次接触的话,可能有些不太好理解.这里作些解释,但是就本书的使用来说,完全可以忽略,而不影响我们做要做的事情.

解释型(interpreting)意味着,写好的、人可读代码直接一行一行地被执行. 而在 C/C++、Java 等语言里,编译器(Compiler)\index{编译器, Compiler}先把相关的代码文件编译(compile)
成可执行的、机器可读而人不可读的二进制指令文件,然后被执行或解释。

\end{document} 
