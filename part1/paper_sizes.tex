\documentclass[main.tex]{subfiles}
\begin{document}

210 × 297	8.27 × 11.69

早在1786年,一位德国的科学家格奥尔格·克里斯托夫·利希滕贝格(Georg Christoph Lichtenberg, 1742年7月1日-1799年2月24日) 就发现宽高比为
2
的矩形具有许多特点[1]。在20世纪初期Walter Porstmann将此概念应用在一系列纸张尺寸的制定,并且在1922年被制定为德国标准(DIN 476)。随着各国逐渐采用,后来此标准被定为国际标准(ISO 216)。
国际标准的纸张尺寸可分为A、B、和C三种系列。此标准的特色是纸张尺寸的宽高比均为
2
(约为1.4142)。同系列但不同尺寸的纸张,其几何比例相同,因此可以直接缩放影印而不会造成纸面图案有边缘裁切的问题。
A系列[编辑]
A系列的制定基础首先是求取一张宽高比为
2
且面积为1平方米(m²)的纸张。因此这张纸的宽长分别为 841 毫米和 1189 毫米(宽高比为
2
:1),并且编号为A0。若将A0纸张的长边对切为二,则得到两张A1的纸张,其宽高均为 594 毫米和841 毫米。依此方式继续将A1纸张对切,则可以依序得到A2、A3、A4等等纸张尺寸。在制定标准时,尺寸均以整数为准,因此对切的纸张尺寸若带有小数(小于 1 毫米)则会舍入计算。

Format	A 系列	B 系列	C 系列
尺寸	毫米 × 毫米	英寸 × 英寸	毫米 × 毫米	英寸 × 英寸	毫米 × 毫米	英寸 × 英寸
0	841 × 1189	33.11 × 46.81	1000 × 1414	39.37 × 55.67	917 × 1297	36.10 × 51.06
1	594 × 841	23.39 × 33.11	707 × 1000	27.83 × 39.37	648 × 917	25.51 × 36.10
2	420 × 594	16.54 × 23.39	500 × 707	19.69 × 27.83	458 × 648	18.03 × 25.51
3	297 × 420	11.69 × 16.54	353 × 500	13.90 × 19.69	324 × 458	12.76 × 18.03
4	210 × 297	8.27 × 11.69	250 × 353	9.84 × 13.90	229 × 324	9.02 × 12.76
5	148 × 210	5.83 × 8.27	176 × 250	6.93 × 9.84	162 × 229	6.38 × 9.02
6	105 × 148	4.13 × 5.83	125 × 176	4.92 × 6.93	114 × 162	4.49 × 6.38
7	74 × 105	2.91 × 4.13	88 × 125	3.46 × 4.92	81 × 114	3.19 × 4.49
8	52 × 74	2.05 × 2.91	62 × 88	2.44 × 3.46	57 × 81	2.24 × 3.19
9	37 × 52	1.46 × 2.05	44 × 62	1.73 × 2.44	40 × 57	1.57 × 2.24
10	26 × 37	1.02 × 1.46	31 × 44	1.22 × 1.73	28 × 40	1.10 × 1.57


186mm x 240mm
787 x 1092 1/32 
\newpage
\end{document} 
