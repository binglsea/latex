\documentclass[zihao=-4,heading=true,a4paper,twoside,openany]{ctexbook}
%\documentclass[hep,10pt]{hepbook}
%\usepackage{HEP}
\usepackage[utf8]{inputenc}
\usepackage[OT2,OT1]{fontenc}
%\pagestyle{empty}
\input HZcaption
\input command

% packages for page layout
\usepackage{geometry}
\geometry{a4paper,total={171.8mm,246.2mm},left=19.1mm,top=25.4mm,}
%\usepackage[body={132mm,198mm},centering]{geometry}

\usepackage{xltxtra,fontspec,xunicode}

% package for colors
\usepackage{xcolor}
\usepackage[contents=征求意见稿22.05]{background}

% packages for fonts
\usepackage{amsmath}
\usepackage{amssymb}
\usepackage{bm}

% packages for graphics
\usepackage{graphicx}
\graphicspath{ {code/} }
\usepackage{wrapfig}
\usepackage{subcaption}
\usepackage{capt-of}
\usepackage{cutwin}
\usepackage{animate}
\usepackage{tikz}

\usepackage{fancyhdr}
%\pagestyle{fancy}
\fancyhead{}
\fancyhead[LE,RO]{\leftmark}
\fancyfoot{}
\fancyfoot[LE,RO]{\thepage}
%\fancyfoot[LO,RE]{更新至\today}

% package for quotes
\usepackage{csquotes}

% redefine itemize
\usepackage{enumitem}% http://ctan.org/pkg/enumitem
\setlist{nosep}

% package for hyperlinks
\usepackage{hyperref}
\hypersetup{colorlinks=true, linkcolor=black}

\usepackage[T1]{fontenc}

% packages for footnotes
\renewcommand{\thefootnote}{\fnsymbol{footnote}}
\usepackage{perpage}
\MakePerPage{footnote} 

% packages for bibliography
%\bibliographystyle{alpha}
\usepackage[backend=biber,style=gb7714-2015,gbpub=false,sorting=none]{biblatex}
\addbibresource{./ref/ref.bib}

\usepackage{imakeidx}
\makeindex

%\usepackage{svg}
\usepackage{listings}
\lstset{ % 编程代码整体设置
	basicstyle=\sffamily,
	keywordstyle=\bfseries,
	commentstyle=\rmfamily\itshape,
	stringstyle=\ttfamily,
	numbers=left,
	escapechar=`,
	numberstyle=\footnotesize
}

\usepackage{setspace}

% packages for definitions, theorems, proofs.
\usepackage{amsthm}
\newtheorem{definition}{定义}
\newtheorem*{definition*}{定义}
\newtheorem{theorem}{定理}
\newtheorem*{theorem*}{定理}
\newtheorem{lemma}{引理}
\newtheorem{corollary}{推论}[theorem]
\newtheorem{example}{例}
\let\oldproof\proof
\renewcommand{\proof}{\color{gray}\oldproof}
% counter controlling
\usepackage{chngcntr}
\counterwithin{section}{chapter}
\counterwithin{figure}{section}
\counterwithin{equation}{section}
\counterwithin{definition}{section}
\counterwithin{theorem}{section}
\counterwithin{lemma}{section}
\counterwithin{example}{section}
\setcounter{tocdepth}{1}

% package for subfiles
\usepackage{subfiles}

%\usepackage{datetime2}
\usepackage{verbatim} 

\usepackage{titlesec}
\titleformat{\section}[block]{\filcenter\scshape\bfseries\Large}{$\S$\,\thesection.}{0.5em}{}

\frontmatter


\title{\begin{kaishu}数学:编程与作图\end{kaishu}\\
	Python、PostScript 和 SVG 的浅入}
\author{\kaishu 龙\,  冰}
\date{}
%=====================================================================================================================
\begin{document}
\begin{titlepage}
	\maketitle
\end{titlepage}

\pagenumbering{roman}
\setcounter{page}{5}%
\subfile{preface}%\cdp
\markboth{目\quad 录}{目\quad 录}
\tableofcontents

\mainmatter
\pagenumbering{arabic}

\part{\begin{lishu}数学:表达美和创造美\end{lishu}}\label{partI}
  \subfile{part1.tex}
\chapter{数学:表达美和创造美}\label{ch1}
  \subfile{chap1/ch1.tex}
\section{万物皆数?}\label{ch1.1}
  \subfile{chap1/1.1_nat_num.tex}
\section{无限}\label{ch1.2}
  \subfile{chap1/1.2_infinite.tex}
\section{数的进位制}
  \subfile{chap1/bases.tex}
\section{偏爱有理数}
  \subfile{chap1/rationals.tex}
\section{尺寸}
  \subfile{chap1/sizes.tex}
\section{汉字的数字化}
  \subfile{chap1/chars.tex}
\section{颜色}
  \subfile{chap1/colors.tex}

\chapter{图: 更优美地表达数学}\label{ch2}
\section{作图证明}
\subfile{chap2/graphs.tex}

\part{\begin{lishu}Python 编程\end{lishu}}\label{partII}
\chapter{Python 入门}\label{chap3}
\section{关于 Python}\subfile{chap3/py_intro.tex}\label{chap:3.1}
\section{安装 Python 开发环境软件集 Anaconda}
 \subfile{chap3/py_install.tex}\label{chap:3.2}

\section{初识 Spyder}
 \subfile{chap3/spyder1.tex}\label{chap:3.3}
\section{初识 Jupyter}
\subfile{chap3/jupyter1.tex}\label{chap:3.4}

\section{常用进位制}\label{chap:3.5}
\subfile{chap3/nadic.tex}

\chapter{Python 应用}
\section{Jieba}
\subfile{chap4/jieba.tex}

\part{\begin{lishu}PostScript 编程作图\end{lishu}}
\section{Some graphs}
\subfile{ps/3.1.tex}

\section{A List}
\subfile{ref/reference.tex}

\part{\begin{lishu}Python 作图\end{lishu}}

\backmatter

\printbibliography

\begin{small}
	\thispagestyle{empty}
	\onecolumn
	\addcontentsline{toc}{chapter}{\indexname}
	\printindex
\end{small}
\vfill
\eject

\end{document}
