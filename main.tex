\documentclass[zihao=-4,heading=true,a4paper,twoside,openany]{ctexbook}
%\documentclass[hep,10pt]{hepbook}
%\usepackage{HEP}
\usepackage[utf8]{inputenc}
%\pagestyle{empty}
\input HZcaption
\input command

% packages for page layout
\usepackage{geometry}
\geometry{a4paper,total={171.8mm,246.2mm},left=19.1mm,top=25.4mm,}
%\usepackage[body={132mm,198mm},centering]{geometry}

\usepackage{xltxtra,fontspec,xunicode}

% package for colors
\usepackage{xcolor}
\usepackage[contents=征求意见稿]{background}

% packages for fonts
\usepackage{amsmath}
\usepackage{amssymb}
\usepackage{bm}

% packages for graphics
\usepackage{graphicx}
\graphicspath{ {figs/} }
\usepackage{wrapfig}
\usepackage{subcaption}
\usepackage{capt-of}
\usepackage{cutwin}
\usepackage{animate}
\usepackage{tikz}

\usepackage{fancyhdr}
%\pagestyle{fancy}
\fancyhead{}
\fancyhead[LE,RO]{\leftmark}
\fancyfoot{}
\fancyfoot[LE,RO]{\thepage}
%\fancyfoot[LO,RE]{更新至\today}

% package for quotes
\usepackage{csquotes}

% redefine itemize
\usepackage{enumitem}% http://ctan.org/pkg/enumitem
\setlist{nosep}

% package for hyperlinks
\usepackage{hyperref}
\hypersetup{colorlinks=true, linkcolor=black}

\usepackage[T1]{fontenc}

% packages for footnotes
\renewcommand{\thefootnote}{\fnsymbol{footnote}}
\usepackage{perpage}
\MakePerPage{footnote} 

% packages for bibliography
\usepackage[backend=biber,style=gb7714-2015,gbpub=false,sorting=none]{biblatex}
\addbibresource{./ref/ref.bib}

\usepackage{imakeidx}
\makeindex

%\usepackage{svg}
\usepackage{listings}

\usepackage{setspace}

% packages for definitions, theorems, proofs.
\usepackage{amsthm}
\newtheorem{definition}{定义}
\newtheorem*{definition*}{定义}
\newtheorem{theorem}{定理}
\newtheorem*{theorem*}{定理}
\newtheorem{lemma}{引理}
\newtheorem{corollary}{推论}[theorem]
\newtheorem{example}{例}
\let\oldproof\proof
\renewcommand{\proof}{\color{gray}\oldproof}
% counter controlling
\usepackage{chngcntr}
\counterwithin{figure}{section}
\counterwithin{equation}{section}
\counterwithin{definition}{section}
\counterwithin{theorem}{section}
\counterwithin{lemma}{section}
\counterwithin{example}{section}
\counterwithin{section}{part}
\setcounter{tocdepth}{1}

% package for subfiles
\usepackage{subfiles}

%\usepackage{datetime2}

\frontmatter

\title{\begin{kaishu}数学:编程与作图\end{kaishu}\\
	Python、PostScript 和 SVG 的浅入}
\author{\kaishu 龙\,  冰}
\date{}
%=====================================================================================================================
\begin{document}
\begin{titlepage}
	\maketitle
\end{titlepage}

\pagenumbering{roman}
\setcounter{page}{5}%
\documentclass[main.tex]{subfiles}
\begin{document}
过去,流行“学数学只需一张纸和一支笔”的说法。上世纪作者考入四川大学数学系七七级,1958 年设立的计算机专业还是数学系的一部分,
分出去成立计算机系并壮大成计算机(及软件)学院是以后的事了。
计算机程序设计是数学系的必修课,但由于各种条件的限制,课程的开设成了摆设,猜想大多数同学认为是浪费时间。
我们学习的编程语言是ALGOL 60,编程只能写在作业本上。
学校有一台巨大的当时比较高级的计算机放在一大房间里精心维护着。
工作人员穿着白大掛,地板打着蜡。
可是我们平时没有机会接触。只是到了期末考试,才有机会进机房。
开卷考试题是解二次方程 $ax^2+bx+c = 0$。
由专门操作人员在纸带上打孔,我们也不懂打孔,只听机器卡嚓卡嚓响,我们就都稀里糊涂地通过了。
由于计算机技术的巨大进步,任何一台个人计算机都会拥有比那台宝贝大不是知多少倍的计算能力。在个人计算机上编程以及当时不可能的作图是件再容易不过的事儿了。另一方面,个人计算机和作为个人计算机的笔记本电脑也已经相当普及,用途也是过去不可想象的广泛,计算机已经成了很多人工作甚至生活不可缺少的一部分。
时代不同了,对数学来说,“一张纸和笔一支笔”的说法肯定是过时了,至少得加上一台个人计算机。对于数学工作者和数学爱好者来说,计算机除了被普通人用作上互联网和玩游戏的工具外,还应当成为数学学习和数学研究的得力助手。

事实上,数学工作者在计算机上用\TeX, \AmS-\TeX, \LaTeX 等排版自己含大量数学符号的文章和书籍,从上世纪八、九十年代开始就逐渐普及。本书亦采用\LaTeX 排版, 并使用 github 对本书的文本和源代码的更新进行云端管理控制。
计算机已经起到了纸和笔的作用,但排版仅属于广义上的计算。计算机应当在一般意义和更广意义上的“计算"上,都在数学学习和数学研究中扮演更精彩的角色。

在数值分析以及工程计算数学应用方面,普遍使用商业软件 MATLAB${}^{\textregistered}$。在统计学方面,除了昂贵的商业软件外,也有广受欢迎的开源 R(作为商用 S-PLUS${}^{\textregistered}$ 的替代)、PSPP(作为商用 SPSS${}^{\textregistered}$ 的替代)等等。这些软件都有针对性特别强的编程语言。这些针对性使得这些语言对于纯数学来讲,过于着眼浮点数\index{浮点数}的运算而忽视了浮点数与整数和有理数在概念上的根本区别。这种区别对作图来说常常无关紧要,但是对一大类数学计算(比如说数论)来说区别是至关重要的。换句话说,没有这类概念的区别,不仅美不存在了,而且涉及整数和有理数的数学问题都无从谈起。1 和 1.0 是否相等是件大事!1 是整数,1.0 是浮点数。计算机对它们储存和运算的处理方式完全不一样。

普通中学数学教育足以


本书的程序代码可从 https://github.com/bingmath 下载(确切的网址待定)。


\begin{flushright}
	\centering
{\kaishu 龙\,  冰}\\
2021年10月
\end{flushright}
\end{document}

%\cdp
\markboth{目\quad 录}{目\quad 录}
\tableofcontents

\mainmatter
\pagenumbering{arabic}

\part{\begin{lishu}数学:表达美和创造美\end{lishu}}\label{part:表达美和创造美}
\chapter{数学:表达美和创造美}\label{chap:1}
\subfile{chap1/ch1.tex}
\section{数}\subfile{chap1/numbers.tex}
\section{数的进位制}\subfile{chap1/bases.tex}
\section{尺寸}\subfile{chap1/sizes.tex}
\section{颜色}\subfile{chap1/colors.tex}
\section{美离不开图}\subfile{chap1/graphs.tex}

\part{\begin{lishu}Python 编程\end{lishu}}
\chapter{Python 入门}\label{chap2}
\section{关于 Python}\subfile{chap2/py_intro.tex}\label{chap:2.1}
\section{安装 Python 开发环境集软件 Anaconda}
 \subfile{chap2/py_install.tex}\label{chap:2.2}

\section{初识 Spyder}
 \subfile{chap2/spyder1.tex}\label{chap:2.3}
\section{初识 Jupyter}
\subfile{chap2/jupyter1.tex}\label{chap:2.4}

\section{常用进位制}\label{sec:I.2}
\subfile{chap2/nadic.tex}

\chapter{Python 应用}
\section{Jieba}
\subfile{chap3/jieba.tex}

\part{\begin{lishu}PostScript 编程作图\end{lishu}}
\section{Some graphs}
\subfile{ps/3.1.tex}

\section{A List}
\subfile{ref/reference.tex}

\part{\begin{lishu}Python 作图\end{lishu}}

\backmatter

\printbibliography

\begin{small}
	\thispagestyle{empty}
	\onecolumn
	\addcontentsline{toc}{chapter}{\indexname}
	\printindex
\end{small}
\vfill
\eject

\end{document}
