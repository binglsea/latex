\documentclass[zihao=-4,linespread=1.5,heading=true,a4paper,twoside]{ctexart}
\usepackage[utf8]{inputenc}
\pagestyle{empty}
% packages for page layout
\usepackage{geometry}
\geometry{
a4paper,
total={171.8mm,246.2mm},
left=19.1mm,
top=25.4mm,
}

% package for colors
\usepackage{xcolor}

% packages for fonts
\usepackage{amsmath}
\usepackage{amssymb}
\usepackage{bm}

% packages for graphics
\usepackage{graphicx}
\graphicspath{ {./images/}{./images/} }
\usepackage{wrapfig}
\usepackage{subcaption}
\usepackage{capt-of}
\usepackage{cutwin}

% packages for page layout
\usepackage{geometry}
\geometry{
a4paper,
total={171.8mm,246.2mm},
left=19.1mm,
top=25.4mm,
}
\usepackage{fancyhdr}
\pagestyle{fancy}
\fancyhead{}
\fancyhead[LE,RO]{\leftmark}
\fancyfoot{}
\fancyfoot[LE,RO]{\thepage}
%\fancyfoot[LO,RE]{更新至\today}

% package for quotes
\usepackage{csquotes}

% redefine itemize
\usepackage{enumitem}% http://ctan.org/pkg/enumitem
\setlist{nosep}



% package for hyperlinks
\usepackage{hyperref}
\hypersetup{colorlinks=true}

% packages for footnotes
\renewcommand{\thefootnote}{\fnsymbol{footnote}}
\usepackage{perpage}
\MakePerPage{footnote} 

% packages for bibliography
\usepackage[
backend=biber,
style=gb7714-2015,
gbpub=false,
sorting=none
]{biblatex}
\addbibresource{./ref/ref.bib}

% packages for definitions, theorems, proofs.
\usepackage{amsthm}
\newtheorem{definition}{定义}
\newtheorem*{definition*}{定义}
\newtheorem{theorem}{定理}
\newtheorem*{theorem*}{定理}
\newtheorem{lemma}{引理}
\newtheorem{corollary}{推论}[theorem]
\newtheorem{example}{例}
\let\oldproof\proof
\renewcommand{\proof}{\color{gray}\oldproof}
% counter controlling
\usepackage{chngcntr}
\counterwithin{figure}{section}
\counterwithin{equation}{section}
\counterwithin{definition}{section}
\counterwithin{theorem}{section}
\counterwithin{lemma}{section}
\counterwithin{example}{section}
\counterwithin{section}{part}
\setcounter{tocdepth}{1}

% package for subfiles
\usepackage{subfiles}

\usepackage{datetime2}

\title{\begin{kaishu}数学:编程与作图\end{kaishu}\\
	Python 与 PostScript 的浅入深出}
\author{\kaishu 龙\,  冰\, \, 钟尔杰\,\,  等}
%=====================================================================================================================
\begin{document}
\begin{titlepage}
	\maketitle
\end{titlepage}

\part*{\kaishu 前\, 言}\label{sec:preface}
\subfile{preface.tex}

\newpage\tableofcontents

\newpage\part{开源软件安装}
\subfile{Part I/I.1.tex}

\newpage\part{python 程序设计}
\section{待定}\label{sec:II.1}
\subfile{py/2.1.tex}

\section{常用进位制}\label{sec:II.2}
\subfile{part1/nadic.tex}

%\section{内积空间与赋范向量空间}\label{sec:II.3}
%\subfile{Part II/II.3.tex}

%\section{线性变换}\label{sec:II.4}
%\subsection{线性变换的定义和基本性质}\label{sec:II.4.1}
%\subfile{Part II/II.4.1.tex}

%\subsection{线性变换的坐标矩阵}\label{sec:II.4.2}
%\subfile{Part II/II.4.2.tex}

%\subsection{线性变换的转置}\label{sec:II.4.3}
%\subfile{Part II/II.4.3.tex}

%\section{基变换与坐标变换}\label{sec:II.5}
%\subfile{Part II/II.5.tex}

%\section{内积空间上的线性算符}\label{sec:II.6}
%\subfile{Part II/II.6.tex}

%\section{线性算符的行列式、迹和特征值}\label{sec:II.7}
%\subfile{Part II/II.7.tex}

%\section{正规算符及其谱分解}\label{sec:II.8}
%\subfile{Part II/II.8.tex}

%\section{欧几里得空间}
%\subfile{Part II/II.9.tex}

%\section{向量函数及其图像}
%\subfile{Part II/II.10.tex}

%\section{向量函数的极限与连续性}
%\subfile{Part II/II.11.tex}

%\section{向量函数的微分与导数}
%\subfile{Part II/II.12.tex}

%\section{曲线、曲面和积分定理}
%\subfile{Part II/II.13.tex}

\newpage
\part{PostScript 编程制图}
\section{Some graphs}
\subfile{ps/3.1.tex}

%\section{物体的运动}
%\subfile{Part III/III.2.tex}

%\section{物体的形变}
%\subfile{Part III/III.3.tex}

%\section{物质描述与空间描述}
%\subfile{Part III/III.4.tex}

%\section{应变率张量}
%\subfile{Part III/III.5.tex}


%\newpage\part{附录}
%\section{线性代数部分定理的证明}\label{sec:VI.1}
%\subfile{Part VI/VI.1.tex}

%\section{向量函数微积分部分的证明}\label{sec:VI.2}
%\subsection{$\mathbb{R}^n$空间上的一些拓扑概念}
%\subfile{Part VI/VI.2.1.tex}

%\subsection{向量函数可微分的必要条件与充分条件}
%\subfile{Part VI/VI.2.2.tex}

%\subsection{复合函数求导的链式法则}
%\subfile{Part VI/VI.2.3.tex}

%\subsection{反函数定理和隐函数定理}
%\subfile{Part VI/VI.2.4.tex}

%\subsection{等距变换的表示定理}
%\subfile{Part VI/VI.2.5.tex}

%\section{曲线坐标系}
%\subfile{Part VI/VI.4.tex}



\newpage\part*{参考文献}
\printbibliography[heading=none]
\end{document}