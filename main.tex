\documentclass[zihao=-4,heading=true,a4paper,twoside,openany]{ctexbook}
\usepackage[utf8]{inputenc}
\pagestyle{empty}

% packages for page layout
\usepackage{geometry}
\geometry{a4paper,total={171.8mm,246.2mm},left=19.1mm,top=25.4mm,}

% package for colors
\usepackage{xcolor}
\usepackage[contents=征求意见稿]{background}

% packages for fonts
\usepackage{amsmath}
\usepackage{amssymb}
\usepackage{bm}

% packages for graphics
\usepackage{graphicx}
\graphicspath{ {./images/} }
\usepackage{wrapfig}
\usepackage{subcaption}
\usepackage{capt-of}
\usepackage{cutwin}
\usepackage{animate}
\usepackage{tikz}

\usepackage{fancyhdr}
\pagestyle{fancy}
\fancyhead{}
\fancyhead[LE,RO]{\leftmark}
\fancyfoot{}
\fancyfoot[LE,RO]{\thepage}
%\fancyfoot[LO,RE]{更新至\today}

% package for quotes
\usepackage{csquotes}

% redefine itemize
\usepackage{enumitem}% http://ctan.org/pkg/enumitem
\setlist{nosep}

% package for hyperlinks
\usepackage{hyperref}
\hypersetup{colorlinks=true, linkcolor=black}

% packages for footnotes
\renewcommand{\thefootnote}{\fnsymbol{footnote}}
\usepackage{perpage}
\MakePerPage{footnote} 

% packages for bibliography
\usepackage[backend=biber,style=gb7714-2015,gbpub=false,sorting=none]{biblatex}
\addbibresource{./ref/ref.bib}

\usepackage{imakeidx}
\makeindex

\usepackage{svg}
\usepackage{listings}

\usepackage{setspace}

% packages for definitions, theorems, proofs.
\usepackage{amsthm}
\newtheorem{definition}{定义}
\newtheorem*{definition*}{定义}
\newtheorem{theorem}{定理}
\newtheorem*{theorem*}{定理}
\newtheorem{lemma}{引理}
\newtheorem{corollary}{推论}[theorem]
\newtheorem{example}{例}
\let\oldproof\proof
\renewcommand{\proof}{\color{gray}\oldproof}
% counter controlling
\usepackage{chngcntr}
\counterwithin{figure}{section}
\counterwithin{equation}{section}
\counterwithin{definition}{section}
\counterwithin{theorem}{section}
\counterwithin{lemma}{section}
\counterwithin{example}{section}
\counterwithin{section}{part}
\setcounter{tocdepth}{1}

% package for subfiles
\usepackage{subfiles}

%\usepackage{datetime2}


\title{\begin{kaishu}数学:编程与作图\end{kaishu}\\
	Python、PostScript 和 SVG 的浅入}
\author{\kaishu 龙\,  冰}
\date{}
%=====================================================================================================================
\begin{document}
\begin{titlepage}
	\maketitle
\end{titlepage}
\chapter*{\kaishu 前\, 言}\label{chap:preface}
\subfile{preface.tex}

\tableofcontents

\part{\begin{lishu}数学:表达美和创造美\end{lishu}}\label{part:表达美和创造美}
\chapter{数学:表达美和创造美}\label{chap:1}
\begin{flushright}
	\begin{kaishu}
	感人的歌声留给人的记忆是长远的。\\
	\end{kaishu}
- 摘自中学语文课文
\end{flushright}
\section{数}\subfile{math/numbers.tex}
\section{数的进位制}\subfile{math/bases.tex}
\section{尺寸}\subfile{math/sizes.tex}
\section{颜色}\subfile{math/colors.tex}
\section{美离不开图}\subfile{math/graphs.tex}

\part{\begin{lishu}Python 编程\end{lishu}}
\chapter{Python 入门}\label{chap2}
\section{关于 Python}
\subfile{py/python_intro_2_1.tex}\label{chap:2.1}
\section{选择 Python 开发环境,安装 Spyder IDE 软件}
 \subfile{py/install_python_2_2.tex}\label{chap:2.2}

\section{初识 Spyder}
 \subfile{py/first_py.tex}\label{chap:2.2}

\section{常用进位制}\label{sec:I.2}
\subfile{py/nadic.tex}


\part{\begin{lishu}PostScript 编程作图\end{lishu}}
\section{Some graphs}
\subfile{ps/3.1.tex}

\section{A List}
\subfile{ref/reference.tex}

\part{\begin{lishu}Python 作图\end{lishu}}


\printindex
\printbibliography


\end{document}
