\documentclass[main.tex]{subfiles}
\begin{document}

\begin{flushright}
	\begin{kaishu}
		不犯错者一事无成.\\
	\end{kaishu}
{\fontencoding{OT2}\selectfont Ne oxibaet{}s{ya} tot, kto niqego ne lelaet.}\\
%Как говорят русские: Не ошибается тот, кто ничего не делает
俄国彦语
\end{flushright}

Python 语言的最初设计者和主要架构师吉多·范罗苏姆 (Guido van Rossum), 荷兰人, 1982 年在阿姆斯特丹大学获得数学和计算机科学硕士学位.
据说 1989 年的圣诞节期间,为了打发时间而开发一个新的脚本\index{脚本, Script}解释器\index{解释器, interpreter}(script interpreter),以替代使用Unix shell和C语言进行系统管理.
后来 Python 成了一种广泛使用的解释型高级通用编程语言,支持多种编程范型,包括函数式、指令式、结构化、面向对象和反射式编程.
它拥有动态类型系统和垃圾回收功能,能够自动管理内存使用,并且其本身拥有一个巨大而广泛的标准库.
这里提到的名词术语,每个都可单独成书论述.如果初次接触的话,可能不太好理解.
这里作些极有限度的解释,但是就本书的使用来说,完全可以忽略,而不影响我们做要做的事情.

解释型(interpreting)意味着,写好的、人可读代码直接一行一行地被执行.而在 C、C++、C\# 和 Java 等语言里,须用编译器(compiler)\index{编译器, Compiler} 先把相关的代码文件编译(compile)
成可执行的、机器可读而人不可读的二进制指令文件,然后被执行或解释.在静态类型语言中,变量的类型须事先申明.一经申明,不可变更.比如一个已申明为浮点数的变量,不可以改变初衷而赋予它字符串的值.有编译器的话,变量的类型会在翻译过程中被验证.
而在动态类型的语言中,变量类型在编程中可以改变.一个数值变量可以不先确定它是整数或分数,或者是表达它的字符串

\end{document} 
