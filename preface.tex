\chapter*{序\quad 言}

\setcounter{page}{1}
\thispagestyle{empty}
\markboth{序\quad 言}{序\quad 言}
\addtocontents{toc}{\vspace{3mm}\leftline{\bf\large 序言}\vspace{3mm}}

过去,流行“学数学只需一张纸和一支笔”的说法.上世纪作者考入四川大学数学系七七级, 1958 年设立的计算机专业还是数学系的一部分,
分出去成立计算机系并壮大成计算机(及软件)学院是以后的事了.
计算机程序设计是数学系的必修课,但由于各种条件的限制,课程的开设成了摆设,猜想大多数同学认为是浪费时间.
我们学习的编程语言是ALGOL 60,编程只能写在作业本上.
学校有一台巨大的当时比较高级的计算机放在一大房间里精心维护着.
工作人员穿着白大掛,地板打着蜡.
可是我们平时没有机会接触.只是到了期末考试,才有机会进机房.
开卷考试题是解二次方程 $ax^2+bx+c = 0$.
由专门操作人员在纸带上打孔,我们也不懂打孔,只听机器卡嚓卡嚓响,我们就都稀里糊涂地通过了.
由于计算机技术的巨大进步,任何一台个人计算机都会拥有比那台宝贝大不是知多少倍的计算能力.在个人计算机上编程以及当时不可能的作图是件再容易不过的事儿了.个人计算机和作为个人计算机的笔记本电脑也已经相当普及,用途也是过去不可想象的广泛,计算机已经成了很多人工作甚至生活不可缺少的一部分.
时代不同了,对数学来说,“一张纸和笔一支笔”的说法肯定是过时了,至少得加上一台个人计算机.对于数学工作者和数学爱好者来说,计算机除了被普通人用作上互联网和玩游戏的工具外,还应当成为数学学习和数学研究的得力助手.

事实上,在计算机上用\TeX,  \LaTeX 等排版含大量数学符号的文章和书籍,从上世纪八、九十年代开始就逐渐普及.本书亦采用\LaTeX 排版, 并使用 github 对本书的源文件和源代码的更新进行云端管理控制.本书讨论的作图可直接或以适当的格式输出用于\LaTeX, 但我们不打算讨论\LaTeX.
计算机已经扩展了纸和笔的作用,但排版仅属于广义上的计算.计算机应当在一般意义和更广意义上的“计算”上,都在数学学习和数学研究中扮演更精彩的角色.
以此为目标, 本书针对大学数学专业而编写.
可以理解, “深入浅出”使许多优秀教科书受读者欢迎. 
本书则尝试“浅入深出”.入门的背景很低,
普通中学的数学训练足以为更一般的非数学专业的数学爱好者打开一扇门, 使他们不太费力气地学习本书的基本内容. 编程要用到的英语, 仅仅是少量简单的词汇, 没有什么难度.
这是浅入,如$\S$\ref{chI.1.1} 标题所说, 一切从0开始.但是,有不少编程例子和应用取自大学数学课程, 也许读者暂时还没有学到或不会选修. 对这些例子和应用,
读者可以只是浏览或者暂时略去,在一些暂时比较深的地方暂时退出来.
这就是我们设想的深出.
但我们提供一个参考方向,给未来创造一个机会.
读者可在条件成熟的时候回头进一步地深入学习.
对于一些完全可以略过的部分,我们用较小的字体排版.

数学编程与作图,我们选用什么编程语言?
第一、其开发软件必须是开源和免费的.
数学方面有著名的商业软件 Mathematica${}^{\textregistered}$ 和 Maple${}^{\textregistered}$.
在数值分析以及工程计算数学应用方面,商业软件 MATLAB${}^{\textregistered}$的使用普遍受欢迎.
它们都有自己定制的个性化编程语言.
在选择编程语言时,自然应当尽量避免被必须与之匹配的商业软件套牢.
付费不必要, 而且像过去MATLAB${}^{\textregistered}$ 突然某一天
在一些范围内被禁止使用的情形更不应困扰我们, 所以开源是必要的.
第二、充分考虑在数学方面的通用性和适用性.
在统计学方面,除了一系列昂贵的商业软件外,开源 R 语言(作为商用 S-PLUS${}^{\textregistered}$ 的替代)及其免费开发平台 R Studio 也广受欢迎.
但是 R 有特别强的针对性,着眼于统计学,计算注重科学计算(MATLAB${}^{\textregistered}$也如此).
比如,过于着眼浮点数\index{浮点数 Floating}的科学计算而忽视了浮点数与整数和有理数在概念上的根本区别.
这种区别对作图来说无关紧要,但是对一大类纯数学计算(比如说数论)来说是至关重要的.没有这类概念的区别,涉及整数和有理数的数学问题都无从谈起.数值分析或计算数学里的计算是科学计算,需要有效地处理误差,广泛应用于工程技术.
虽然本书在一些地方,尤其是在作图方面的计算,也采用科学计算,
但是我们不去考虑误差带来的影响,更不去涉及工程技术.关于数值分析,我们推荐读者参考钟尔杰的《数值分析讲义》\cite{ZhongEr1}和
《数学实验方法》\cite{ZhongEr2}.
除了科学计算,我们需要有效地进行纯数学的精确计算.
除此之外,第三、
我们需要尽量使用一种通用编程语言,使得我们不排除有可能应用所学,做一些不完全是数学方面的事情, 比如文字处理.
我们来看一下目前一些比较常用的软件开发通用编程语言,如 Java、C++、C\# …等,做一件很简单的事也需要讲究做很复杂的事一样的排场,而且这些排场跟数学无关. 
Python 和 Perl 算得上拿得起放得下的通用编程语言.简单的事情一行代码就搞定了,复杂的事情也可让我们直面挑战, 适合数学编程.
实用方面,它们现写即用,不用编译成不可读的二进制机器码.
这就是通常所说的脚本语言\index{脚本语言, scripting language}功能.
我们选择 Python,它的开发环境和工具,可用的包和库,用户普及程度都远胜Perl. Python 的 numpy 包使不少科学计算和纯数学计算变得容易, 它的从 MATLAB${}^{\textregistered}$ 移植过来的 matplotlib 包也使用得其作图功能成一大特色. 而 Perl 在这两方面则逊色. 
Python 近年来越来越多地成为主流的软件开发编程语言,
并成为数据处理,机器学习和人工智能领域等的常用编程语言.
考虑到互联网及网页的普及, 我们也介绍 SVG 作图. PostScript${}^{\textregistered}$ 语言是本书最先被考虑为唯一的作图语言.除了在作图中有趣和可具挑战性格外,它对几何和数理逻辑的学习很有帮助.
许多激光打印机的软件由它写成.许多作图或非作图应用程序也把文档输出
为PostScript${}^{\textregistered}$ 格式. \LaTeX 直接支持PostScript${}^{\textregistered}$图像.
这三种语言的作图可只选一种来学习.

本书不是为计算机专业或软件开发人员而写,但是相信结合本书学习会对他们从另一视角看问题有所启迪.
在 Python 的介绍里,我们把许多跟实用于数学编程无关的重要内容都忽略了.比如,我们常常采用拿来主义,大量使用 Python 内置的或者
输入库里现成的类,但是完全不去讨论如何自己创建类.我们也不去讨论许多如算法和数据结构等重要话题. 虽然我们对 Python 的着眼点是数学工具,
请不要忘记, Python 是开发商业软件的主流通用编程语言之一.
本书更不是为准备和应付考试而写.但仍然对《全国计算机等级考试二级 Python 语 言程序设计考试大纲》有所照顾\footnote{\texttt{http://ncre.neea.edu.cn/html1/report/20122/1392-1.htm}}.

本书的程序代码可从 \verb|https://github.com/longbing-math| 下载.



{\kaishu
	\begin{center}
		\hspace*{88mm}龙\,   冰\\
		\hspace*{88mm}2023年~11月
	\end{center}
}


