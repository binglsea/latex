\chapter*{序\quad 言}

\setcounter{page}{1}
\thispagestyle{empty}
\markboth{序\quad 言}{序\quad 言}
\addtocontents{toc}{\vspace{3mm}\leftline{\bf\large 序言}\vspace{3mm}}

过去,流行“学数学只需一张纸和一支笔”的说法.上世纪作者考入四川大学数学系七七级,1958 年设立的计算机专业还是数学系的一部分,
分出去成立计算机系并壮大成计算机(及软件)学院是以后的事了.
计算机程序设计是数学系的必修课,但由于各种条件的限制,课程的开设成了摆设,猜想大多数同学认为是浪费时间.
我们学习的编程语言是ALGOL 60,编程只能写在作业本上.
学校有一台巨大的当时比较高级的计算机放在一大房间里精心维护着.
工作人员穿着白大掛,地板打着蜡.
可是我们平时没有机会接触.只是到了期末考试,才有机会进机房.
开卷考试题是解二次方程 $ax^2+bx+c = 0$.
由专门操作人员在纸带上打孔,我们也不懂打孔,只听机器卡嚓卡嚓响,我们就都稀里糊涂地通过了.
由于计算机技术的巨大进步,任何一台个人计算机都会拥有比那台宝贝大不是知多少倍的计算能力.在个人计算机上编程以及当时不可能的作图是件再容易不过的事儿了.另一方面,个人计算机和作为个人计算机的笔记本电脑也已经相当普及,用途也是过去不可想象的广泛,计算机已经成了很多人工作甚至生活不可缺少的一部分.
时代不同了,对数学来说,“一张纸和笔一支笔”的说法肯定是过时了,至少得加上一台个人计算机.对于数学工作者和数学爱好者来说,计算机除了被普通人用作上互联网和玩游戏的工具外,还应当成为数学学习和数学研究的得力助手.

事实上,在计算机上用\TeX,  \LaTeX 等排版含大量数学符号的文章和书籍,从上世纪八、九十年代开始就逐渐普及.本书亦采用\LaTeX 排版, 并使用 github 对本书的源文件和源代码的更新进行云端管理控制,但我们不打算讨论这方面的技术.
计算机已经扩展了纸和笔的作用,但排版仅属于广义上的计算.计算机应当在一般意义和更广意义上的“计算”上,都在数学学习和数学研究中扮演更精彩的角色.

事实上,数学方面有著名的商业软件 Mathematica${}^{\textregistered}$ 和 Maple${}^{\textregistered}$.
在数值分析以及工程计算数学应用方面,商业软件 MATLAB${}^{\textregistered}$使用普遍.
它们都有自己的个性化编程语言.
在选择编程语言时,自然应当尽量避免被必须与之匹配的商业软件套牢.
在统计学方面,除了一系列昂贵的商业软件外,开源 R 语言(作为商用 S-PLUS${}^{\textregistered}$ 的替代)及其免费开发平台 R Studio 等也广受欢迎的.
但是 R 有特别强的针对性,除了着眼于统计学外,计算着重科学计算.比如,过于着眼浮点数\index{浮点数 floating}的运算而忽视了浮点数与整数和有理数在概念上的根本区别.
这种区别对作图来说无关紧要,但是对一大类纯数学计算(比如说数论)来说是至关重要的.没有这类概念的区别,涉及整数和有理数的数学问题都无从谈起.
除了科学计算,我们需要有效地进行纯数学计算机.
我们需要尽量使用一种通用编程语言(General-Purpose Programming Language),使得我们不排除有可能应用所学,做一些不完全是数学方面的事情.
另一方面,对一些开发软件的通用编程语言,如 Java、C++、C\# …等,做一件很简单的事也需要讲究做很复杂的事一样的排场,而且这些排场跟数学无关. 
Python 和 Perl 算得上拿得起放得下的通用编程语言.简单的事情一行代码就搞定了,复杂的事情也可让我们直面挑战, 适合数学编程.
我们选择 Python,它的开发环境和工具,可用的包和库,用户普及程度都远胜Perl. Python 的 numpy 包使不少科学计算和纯数学计算变得容易, 它的从 MatLab 移植过来了的 matplotlib 包也使用得其作图功能成一大特色, 而 Perl 没有. 考虑到互联网及网页的普及, 我们也介绍 SVG 作图. PostScript 语言

普通中学数学教育足以



本书不是为软件开发人员而写,但是相信结合本书学习会有所启迪.本书更不是为准备和应付考试而写.但仍然对《全国计算机等级考试二级 Python 语 言程序设计考试大纲》有所照顾\footnote{\texttt{http://ncre.neea.edu.cn/html1/report/20122/1392-1.htm}}.

本书的程序代码可从 \verb|https://github.com/longbing-math| 下载.


{\kaishu
	\begin{center}
		\hspace*{88mm}龙\,   冰\\
		\hspace*{88mm}2021年~10月
	\end{center}
}


