\documentclass[main.tex]{subfiles}
\begin{document}
记得上世纪八十年代初我们在四川大学数学系学习,1958年设立的计算机专业还是数学系的一部分,
分出去成立计算机系并壮大成计算机(及软件)学院是以后的事了。计算机程序设计是数学系的必修课,
但由于各种条件的限制,课程开设成了摆设,猜想大多数同学认为是浪费时间。我们在学习的语言是ALGOL 60,编程只能写在作业本上。
学校一台巨大的当时高级的计算机放在一大房间里精心维护着,工作人员穿着白大掛,地板打着蜡。可是我我们平时没有机会接触。只是到了期末考试,我们才有机会进机房。
开卷考试题是解二次方程 $ax^2+bx+c = 0$。由专门操作人员在纸带上打孔,我们也不懂打孔,只听机器卡嚓卡嚓响,我们就都稀里糊涂地通过了。由于计算机技术的巨大进步,任何一台个人电脑的都会拥有比那台宝贝大不是知多少倍的计算能力。在个人电脑编程也是件再容易不过的事儿了,普通中学数学教育足以

\begin{flushright}
龙\, 冰\,\,\\
2021年10月
\end{flushright}
\end{document}

