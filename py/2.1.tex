\documentclass[main.tex]{subfiles}
% 集合与映射
\begin{document}
集合是近代数学的基本语言。连续介质力学中的大量概念都依赖集合来定义。而集合本身却是难以借助更基础的概念进行定义的“最原始概念”。因此在下述集合的定义只能假定每一位读者都能一致地理解。
\begin{definition}[集合]\label{def:II.1.1}
集合(set)是具有某种特性的事物的整体。构成集合的事物或对象称为元素(element)或成员(member)。集合还必须满足:
\begin{itemize}
    \item 无序性:一个集合中,每个元素的地位是相同的,元素之间是无序的
    \item 互异性:一个集合中,任何两个元素都不相同,即每个元素只出现一次
    \item 确定性:给定一个集合及一个事物,该事物要么属于要么不属于该集合,不允许模棱两可。
\end{itemize}
\end{definition}




\end{document}
