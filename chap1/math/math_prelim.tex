\documentclass[main.tex]{subfiles}
\begin{document}

In this example, several keywords\index{keywords} will be used 
which are important and deserve to appear in the Index\index{Index}.

Terms like generate\index{generate} and some\index{others} will 
also show up. 

https://zh.wikipedia.org/zh-cn/像素

像素,为影像显示的基本单位,译自英文“pixel”,pix是英语单词picture的常用简写,加上英语单词“元素”element,就得到pixel,故“像素”表示“画像元素”之意,有时亦被称为pel(picture element)。每个这样的消息元素不是一个点或者一个方块,而是一个抽象的取样。仔细处理的话,一幅影像中的像素可以在任何尺度上看起来都不像分离的点或者方块;但是在很多情况下,它们采用点或者方块显示。每个像素可有各自的颜色值,可采三原色显示,因而又分成红、绿、蓝三种子像素(RGB色域),或者青、品红、黄和黑(CMYK色域,印刷行业以及打印机中常见)。照片是一个个取样点的集合,在影像没有经过不正确的/有损的压缩或相机镜头合适的前提下,单位面积内的像素越多代表分辨率越高,所显示的影像就会接近于真实物体。


《庄子·杂篇·天下》一尺之棰,日取其半,万世不竭。

一尺长的木棍,每天截掉一半,永远也截不完。
这是庄子的好朋友、名家人物惠施的命题之一。惠施本人没有留下著作,《庄子•天下》保存了他的“历物十事”和二十一个命题。这个命题的意思是:一尺长的木棍,每天截去它的一半,千秋万代也截不完。


\end{document} 
