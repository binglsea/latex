《易经》有二进制的萌芽,但不是真正的二进制计数和计算.
八卦卦符
\begin{itemize}
\item 坤: 阴阴阴
\item 艮: 阴阴阳
\item 坎: 阴阳阴
\item 巽: 阴阳阳
\item 震: 阳阴阴
\item 离: 阳阴阳
\item 兑: 阳阳阴
\item 乾: 阳阳阳
\end{itemize}



\begin{example}
	1/4属于 Cantor 集.
\end{example}
我们来计算一下1/4 的三进位表示.
\begin{align*}
\frac{1}{4} &= \frac{1}{1+3} = \frac{1}{3(1+ \frac{1}{3})} 
  = \frac{1}{3} \cdot \frac{1}{1 - (- \frac{1}{3})} \\
  &	= \frac{1}{3} \cdot \left[ 
1 + \left(- \frac{1}{3}\right)
+  \left(- \frac{1}{3}\right)^2 + \left(- \frac{1}{3}\right)^3 
+ \left(- \frac{1}{3}\right)^4   +\left(- \frac{1}{3}\right)^5  +\cdots\right] \\
  &	= \frac{1}{3} \cdot\Big\{ \left[ 
1 - \left( \frac{1}{3}\right)\right]
+ \left[  \left(\frac{1}{3}\right)^2 - \left(\frac{1}{3}\right)^3 \right]
+ \left[ \left(\frac{1}{3}\right)^4   -\left(\frac{1}{3}\right)^5 \right] +\cdots\Big\}\\
  &	= \frac{1}{3} \cdot \left[ 
 \frac{2}{3}+ \frac{2}{3^3}+ \frac{2}{3^5}+\cdots\right] \\
&=  \frac{2}{3^2}+ \frac{2}{3^4}+ \frac{2}{3^6}+\cdots
\end{align*}
\noindent
它的三进位表示里没有任何位数是 1,由此可见1/4属于 Cantor 集.

\begin{Exercises}
	
	\item 1/5 属于 Cantor 集.(提示:
	$1/5 = \sum_{n = 0}^\infty\frac{2^{2n}}{3^{2n+2}}$.)
	
	
\end{Exercises}


