\documentclass[main.tex]{subfiles}
\begin{document}
\subsection{从自然数到无穷数}
老子说: “道生一,一生二,二生三,
三生万物”\footnote{《道德经》第四十二章: 道生一,一生二,二生三,三生万物. 万物负阴而抱阳,冲气以为和. 人之所恶,唯孤、寡、不毂,而王公以为称。故物或损之而益,或益之而损。人之所教,我亦教之。强梁者不得其死,吾将以为教父.}.
“万物皆数”\footnote{万物皆数的英文原意为Everything is arranged according to number. 
	法国数学家 Mickaël Launay 的《Le grand roman des maths: De la préhistoire à nos jours》一书的中译本书名《万物皆数》则是意译.}
则是古希腊数学家毕达哥拉斯的宇宙观. 他们的观点反映了数在现实世界中的重要性. 数在数学中的地位更是不言而喻.
不过要把数说清楚也不是一件容易的事情. 考虑本书着眼的是数学编程和作图,我们在这里仅适可而止地讨论. 

公理集合论是数理逻辑不可缺少的分支之一,它奠定了整个现代数学的基础. 在公理集合论的宇宙里, 可以认为万物皆数. 
不过这个数可不是毕达哥拉斯认为的有理数\index{有理数, Rational number}, 而是公理集合论里讲的基数.
基数是自然数的推广.

那么,究竟什么是自然数呢?过去,从小学到大学的教科书和专著都说,自然数的集合$\mathbb{N} = \{1, 2, 3, \dots\}$\index{$\mathbb{N}$, 自然数的集合}.
例子有华罗庚的《数论导引》\cite{HuaL}和菲赫金哥尔茨的
《微积分学教程》\cite{FeiH1}.范德瓦尔登《代数学 I》\cite{derWaerden}里关于
皮亚诺公设\index{皮亚诺公设, Peano Postulate}的原来形式也是这样的.
但是现在的说法是,自然数从0开始, 
$$\mathbb{N} = \{0,1, 2, 3, \dots\}.$$
 国际标准化组织(ISO)的标准化文件ISO 80000-2\footnote{较早版本ISO 31-11, 见  https://en.wikipedia.org/wiki/ISO\_31-11}
 规定0是自然数. 较早版本ISO 31-11已经这样规定了. ISO 凭什么来掺和纯数学的概念?其中一个原因是绝大部分计算机编程语言\footnote{R 语言的索引不是基于0的.}里,列表(\texttt{list})\index{列表, \texttt{list}}或者
 组元(\texttt{array)}\index{\texttt{array}, 组元}之类的数据的索引是从0开始,
 专业术语是“基于0的”(0-based)\index{0-based, 基于0的}索引.我们在以后的 Python 编程中会看到.其实更重要的原因还是纯数学界已经改口了.
 所有关于现代公理集合论的著述都如是说\cite{HalmosP}\cite{KelleyJ}\cite{JiangJi}. 各种级别的数学教材也纷纷把0纳入了自然数.我们接下来会讨论整个纯数学是怎样无中生有,从空集$\phi$,也就0开始,构造自然数... 
“道生一,一生二,二生三,三生万物”成了“0生一,一生二,二生三,三生万物”.

人类对0的认识比其它对正整数的认识晚很多. 0的出现起初不那么自然,但现在为什么就很自然了呢?
除了为什么,更重的问题是: 0、1、2、3 ...究竟是什么东西?
读者可能会疑惑,中国人从小学生起就熟背加法口诀和乘法九九表,还问这样低级的问题?
事实上,问题并不低级,也不简单.自然数0、1、2、3 ...在代数或集合论出现之前并不被视为具体东西,而是一种抽象的比较关系.数学上我们通常把俗称的东西叫做元素(Element).
满足一些条件的元素又组成集合(Set).
自然数本来是抽象概念,展示事物共有的一些性质,不是元素.你的我的他/她的每只手和每只脚都有5个指头.但是鸡不同,它的每只爪子只有4个脚趾.一只手/脚的5个指头不能和鸡爪的4个趾1-1 对应,但可以和五角星的5个角1-1 对应.
像5一样,一个个单独的自然数,依靠这种1-1 对应关系而被定义.它常常被不严格地视为一个元素,但是不能说它是一个集合.它是一类有相同个数的有限集合的共同性质.这种定义在朴素集合论里被推广到无穷集合上,叫做势.

Ordinal number, Cardinal, Cardinal number 和 Cardinality在朴素集合论\footnote{有趣的是, Paul Halmos 的《朴素集合论》\cite{HalmosP}讲的是ZFC 公理集合论.}
和公理集合论里的含义是不同的. 
在朴素集合论里,它们是比较和等价关系,是类型.
而在公理集合论里,它们还是明确定义的集合.为了不引起混淆,在公理集合论里,
我们的说序数(Ordinal number)\index{序数, Ordinal number}和基数(Cardinal number, 或 Cardinal)\index{基数, Cardinal},
而在朴素集合论里,我们说势(Cardinality)\index{势, Cardinality}.

康托(Geog Cantor)创立朴素集合论时就有等势的概念.两个集合$A$和$B$,如果其元素有一个 1-1 对应关系,就说$A$和$B$等势,记为$|A|=|B|$.
例如全体正整数的集合和全体正偶数的集合有相同的势,朴素集合论里把这个势记为$\aleph_0$\index{$\aleph_0$, 自然数集的势}.虽然前者远多于后者,但是1-1对应是很容易看出的.这是因为,乘以2可从前者映射到后者,而除以2则可以从后者逆映射回到前者.可以找到无穷多个不同集合,具有势$\aleph_0$.比如平面上坐标为整数的所有的点的集合,也有势$\aleph_0$.任何势,不管有穷还是无穷,它是一类集合所共享的一种等价比较关系.无穷势被视为一种无穷的数.有限集合和无穷集合的区别在于,任何一个无穷集合可以与它的真子集等势.有无穷多个房间的Hilbert 旅馆,客满以后还可以继续接待无穷多的客人: 让已经住下的房客一个一个地依次挪房间.
数学家们发现了无穷集合的许多跟直观一样感觉不一样的性质.朴素集合论让阿猫阿狗进入集合通常也不是问题.问题是,谈一类集合的类(class)时限制太小,集合和真类区分不清楚.再就是早就有的自指(self-reference)逻辑困境以不同形式出现.
在“所有集合的集合”的罗素悖论中,两种情形都有.
之前的关于势的康托悖论和关于朴素集合论序数的Burali-Forti悖论则是集合和真类区分不清楚.

在公理集合论里,数学家们小心翼翼,别说让阿猫阿狗进入集合, 连什么是集合什么是元素,都坚称没有定义.

每个集合都有一个
基数\footnote{公理集合论里没有真正的元素,
	因为每个元素本身就是一个集合, (除空集外)它又含元素.}. 
基数是特殊的序数(Ordinal)\index{序数, Ordinal}.
\footnote{根据选择公理,每个集合有一个良序,从而可找到一个与之有 1-1 }
最简单的序数和基数是自然数.
\subsection{有理数与无理数都不能局限于有限}

$365=10^2+11^2+12^2$.
$365=1^3+2^3+3^3+4^3$.


1 inch = 96px
https://zh.wikipedia.org/zh-cn/像素

像素,为影像显示的基本单位,译自英文“pixel”,pix是英语单词picture的常用简写,加上英语单词“元素”element,就得到pixel,故“像素”表示“画像元素”之意,有时亦被称为pel(picture element)。每个这样的消息元素不是一个点或者一个方块,而是一个抽象的取样。仔细处理的话,一幅影像中的像素可以在任何尺度上看起来都不像分离的点或者方块;但是在很多情况下,它们采用点或者方块显示。每个像素可有各自的颜色值,可采三原色显示,因而又分成红、绿、蓝三种子像素(RGB色域),或者青、品红、黄和黑(CMYK色域,印刷行业以及打印机中常见)。照片是一个个取样点的集合,在影像没有经过不正确的/有损的压缩或相机镜头合适的前提下,单位面积内的像素越多代表分辨率越高,所显示的影像就会接近于真实物体。


《庄子·杂篇·天下》一尺之棰,日取其半,万世不竭。

一尺长的木棍,每天截掉一半,永远也截不完。
这是庄子的好朋友、名家人物惠施的命题之一。惠施本人没有留下著作,《庄子•天下》保存了他的“历物十事”和二十一个命题。这个命题的意思是:一尺长的木棍,每天截去它的一半,千秋万代也截不完。


\end{document} 
