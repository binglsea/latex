\documentclass[main.tex]{subfiles}
\begin{document}

老子说: “道生一,一生二,二生三,
三生万物”\footnote{《道德经》第四十二章: 道生一,一生二,二生三,三生万物. 万物负阴而抱阳,冲气以为和. 人之所恶,唯孤、寡、不毂,而王公以为称。故物或损之而益,或益之而损。人之所教,我亦教之。强梁者不得其死,吾将以为教父.}.
“万物皆数”\footnote{万物皆数的英文原意为Everything is arranged according to number. 
	法国数学家 Mickaël Launay 的《Le grand roman des maths: De la préhistoire à nos jours》一书的中译本书名《万物皆数》则是意译.}
则是古希腊数学家毕达哥拉斯的宇宙观. 他们的观点反映了数在现实世界中的重要性. 数在数学中的地位更是不言而喻.
不过要把数说清楚也不是一件容易的事情, 需要登堂入室元数学
\index{元数学, Metamathematics}, 数学史上的三次危机都跟它有关. 

公理集合论是数理逻辑不可缺少的分支之一,它奠定了整个现代数学的基础. 在公理集合论的宇宙里, 可以认为万物皆数. 
不过这个数可不是毕达哥拉斯认为的有理数\index{有理数, Rational number}, 而是公理集合论里讲的基数 (Cardinal)\index{基数, Cardinal},即朴素
集合论\footnote{有趣的是, Paul Halmos 的《朴素集合论》讲的是ZFC 公理集合论\cite{HalmosP}.}讲的势. 每个集合都有一个
基数\footnote{公理集合论里没有真正的元素,
因为每个元素本身就是一个集合,除空集外它又含元素.}.
最简单的基数是自然数.什么是自然数?过去,从小学到大学的教科书和专著都说,自然数的集合$\mathbb{N} = \{1, 2, 3, \dots\}$\index{$\mathbb{N}$, 自然数的集合}.
例如华罗庚的《数论导引》\cite{HuaL}和菲赫金哥尔茨的
《微积分学教程》\cite{FeiH1},范德瓦尔登《代数学 I》\cite{derWaerden}里关于
皮亚诺公设\index{皮亚诺公设, Peano Postulate}的原来形式也是这样的.
但是现在的说法,自然数从0开始,$\mathbb{N} = \{0,1, 2, 3, \dots\}$.所有现代公理集合论的著作都如是说
Ref
\cite{JiangJi}

%Van der Waerden\cite{derWaerden}.

$365=10^2+11^2+12^2$.
$365=1^3+2^3+3^3+4^3$.


1 inch = 96px
https://zh.wikipedia.org/zh-cn/像素

像素,为影像显示的基本单位,译自英文“pixel”,pix是英语单词picture的常用简写,加上英语单词“元素”element,就得到pixel,故“像素”表示“画像元素”之意,有时亦被称为pel(picture element)。每个这样的消息元素不是一个点或者一个方块,而是一个抽象的取样。仔细处理的话,一幅影像中的像素可以在任何尺度上看起来都不像分离的点或者方块;但是在很多情况下,它们采用点或者方块显示。每个像素可有各自的颜色值,可采三原色显示,因而又分成红、绿、蓝三种子像素(RGB色域),或者青、品红、黄和黑(CMYK色域,印刷行业以及打印机中常见)。照片是一个个取样点的集合,在影像没有经过不正确的/有损的压缩或相机镜头合适的前提下,单位面积内的像素越多代表分辨率越高,所显示的影像就会接近于真实物体。


《庄子·杂篇·天下》一尺之棰,日取其半,万世不竭。

一尺长的木棍,每天截掉一半,永远也截不完。
这是庄子的好朋友、名家人物惠施的命题之一。惠施本人没有留下著作,《庄子•天下》保存了他的“历物十事”和二十一个命题。这个命题的意思是:一尺长的木棍,每天截去它的一半,千秋万代也截不完。


\end{document} 
