\documentclass[main.tex]{subfiles}
\begin{document}
\begin{flushright}
	\begin{kaishu}
		万物皆数。\\
	\end{kaishu}
	- 毕达哥拉斯
\end{flushright}

本章有两个目的.一是为编程作些数学铺垫,是漫谈,算不上是在严格意义上的数学准备.
二是通过一些实际例子说明,在与时俱进的今天,编程即使
在纯数学学习中也不是可有可无.在这些实际例子中,或者数学扮演着的不可缺少的角色,或者数学需要不同的优美表达.但所涉及的计算,手工进行的话太繁重,但用编程则是轻而易举的.本章的中的一些数据的生成和图表的绘制是以后各章的编程例子.
但愿达成一个共识,编程与作图作为一个工具,会成为数学想象的翅膀.

读者可先浏览本章,在以后有关章节中实际编程时,会回到本章有关问题.教师在实际讲课时,也可先讲第\ref{chap2}章,在编程和作图中再结合例子介绍本章内容.
\end{document} 
