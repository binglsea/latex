\documentclass[main.tex]{subfiles}
\begin{document}
现代的二进制记数系统由戈特弗里德·莱布尼茨于1679年设计,在他1703年发表的文章《论只使用符号0和1的二进制算术,兼论其用途及它赋予伏羲所使用的古老图形的意义》[1]出现。与二进制数相关的系统在一些更早的文化中也有出现,包括古埃及、古代中国和古印度。中国的《易经》尤其引起了莱布尼茨的联想。

坤:黑黑黑,卦符阴阴阴
艮:黑黑白,卦符阴阴阳
坎:黑白黑,卦符阴阳阴
巽:黑白白,卦符阴阳阳
震:白黑黑,卦符阳阴阴
离:白黑白,卦符阳阴阳
兑:白白黑,卦符阳阳阴
乾:白白白,卦符阳阳阳

赵钱孙李 周吴郑王 冯陈褚(chǔ)卫 蒋沈韩杨朱秦尤许 何吕施张 孔曹严华 金魏陶姜

《百家姓》各个姓氏排列次序不是以人口数量多少,而是以政治地位为准则。
根据南宋学者王明清考证,“赵钱孙李”之所以是《百家姓》前四姓,是因为百家姓在北宋初年的吴越钱塘地区形成,所以就用当时最重要的家庭姓氏:
赵氏:宋朝皇帝的姓氏。
钱氏:吴越国王的姓氏。
孙氏:吴越国王钱俶正妃的姓氏。
李氏:南唐国王的姓氏。

\end{document} 
