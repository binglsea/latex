\documentclass[main.tex]{subfiles}
\begin{document}
我们要作图,或为写文章写书排版,了解纸张的尺寸自然是有用的。国际标准ISO 216的纸张尺寸可分为A、B、和C三种系列,其制定都跟数学有关。
在中国和欧洲,打印文件纸张的比较常用标准尺寸是A系列中的 A4,即,210毫米\texttt{x}{ }297毫米 ($\approx$ 8.27 英寸 x 11.69 英寸)。
 德国科学家利希滕贝格(Georg Christoph Lichtenberg) 发现高宽比为$\sqrt{2}$
的矩形具有有趣的特点。
设一张矩形的纸宽为$w$,高为$h$,竖着看我们假设$w <h$。横着在高度一半处切裁为两个全等的更小矩形。小矩形不管是顺时针还是逆时针旋转 $90^\circ$,竖起来后设新的宽为$w'$,高为$h'$。则
$w' = h/2$ 且 $h' = w$。若使原矩形与新的小矩形相似,则两个高宽比须相等:
$$\frac{h}{w} = \frac{w}{h/2}.   $$
于是,$h^2= 2 w^2$, 或
 $$h= \sqrt{2} w.$$
这里我们又遇到了一个无理数, $\sqrt{2}$。在工程上,我们较真的程度有赖于技术。
	
\begin{kaishu}习题.\end{kaishu} 执行如下 Python 代码,确认记忆中的$\sqrt{2}$值 $1.4142135623730951\cdots$。
\begin{spacing}{0.8}
	\begin{small}
	\begin{lstlisting}[language=Python]
print("2 的平方根 = ", 2**.5)
\end{lstlisting}
\end{small}
\end{spacing}

A 系列不同开本的纸张从一平方米(1,000,000 平方毫米)的A0开始。设宽为$w_0$ 毫米,则高为
$\sqrt{2}w_0$ 毫米。于是$\sqrt{2}w_0^2 = 1000000$。
由此,
$w_0 = \sqrt{1000000/\sqrt{2}}=1000/\sqrt[4]{2}, 
h_0=\sqrt{2}w_0=1000\cdot\sqrt[4]{2}$。
几行Python代码可算出 A 系列不同开本纸张的尺寸。注意裁开一次页数加倍,所以An是$2^n$开,A4是16开($2^4=16$)。

\vspace{.4cm}
\begin{spacing}{0.8}
	\begin{small}
	\begin{lstlisting}[language=Python]
#%% A系列纸张的理论尺寸
w = 1_000 / 2**.25 #下划线"_"分隔仅为阅读方便,无实际编程作用
h = 1_000 * 2**.25 # 2**.25是2的4次方根即2的0.25次幂 
print("A系列纸张的理论尺寸:")
for i in range(0, 11):
  # {:>3} 表示 3 个字符向右看齐。
  print("{:>3}: {:3.0f} 毫米 x {:4.0f} 毫米".format(
    'A'+str(i), w, h), end='   ')
  # {:6.2f}: 包括小数点,浮点数展示 6 个字符,向右看齐,
  #  小数点后面保留 2 位数
  print(" {:6.2f}毫米 x {:7.2f}毫米".format(w, h), end='  ')
  # 用 int() 丢弃小数部分,宽高仅保留整数部分
  print("   {:3.0f}毫米 x {:4.0f}毫米".format(int(w), int(h)))
  tmp = w  #暂存原宽度备作下一个高度
  w = h/2   #新宽度为原高度的一半
  h = tmp   #新高度为原宽度
\end{lstlisting}
\end{small}
\end{spacing}
\vspace{.4cm}\label{a_paper_py}
输出结果如下:
\vspace{.4cm}\label{a_paper_py}
\begin{spacing}{0.8}
	\begin{small}
\begin{lstlisting}
A系列纸张的理论尺寸:
 A0: 841毫米 x 1189毫米   840.90毫米 x 1189.21毫米     840毫米 x 1189毫米
 A1: 595毫米 x  841毫米   594.60毫米 x  840.90毫米     594毫米 x  840毫米
 A2: 420毫米 x  595毫米   420.45毫米 x  594.60毫米     420毫米 x  594毫米
 A3: 297毫米 x  420毫米   297.30毫米 x  420.45毫米     297毫米 x  420毫米
 A4: 210毫米 x  297毫米   210.22毫米 x  297.30毫米     210毫米 x  297毫米
 A5: 149毫米 x  210毫米   148.65毫米 x  210.22毫米     148毫米 x  210毫米
 A6: 105毫米 x  149毫米   105.11毫米 x  148.65毫米     105毫米 x  148毫米
 A7:  74毫米 x  105毫米    74.33毫米 x  105.11毫米      74毫米 x  105毫米
 A8:  53毫米 x   74毫米    52.56毫米 x   74.33毫米      52毫米 x   74毫米
 A9:  37毫米 x   53毫米    37.16毫米 x   52.56毫米      37毫米 x   52毫米
A10:  26毫米 x   37毫米    26.28毫米 x   37.16毫米      26毫米 x   37毫米\end{lstlisting}
\end{small}
\end{spacing}
\vspace{.4cm}\label{a_paper_py}
国际标准 ISO 216 中毫米仅保留到整数。第一列结果四舍五入显然不符合切裁纸张的实际。第三列用 \texttt{int()}丢弃小数部分,宽高仅保留整数部分,所以应该最符合实际。事实上,除了ISO 216 A1的尺寸是594毫米\texttt{x}{ }841毫米而不同外,第三列的其它都符合国际标准。

\begin{kaishu}习题.\end{kaishu} 在上面的Python代码中的,把\texttt{for} 循环句改为\texttt{while}循环句,并且使输出加一列说明开数。

B系列纸张仍然遵循高宽比约为$\sqrt{2}$的原则,但定义B0的宽为1000毫米(1米),高为1414毫米。

\begin{kaishu}习题.\end{kaishu} 写出 Python 程序,计算 B0 到B10 的开本和理论尺寸。 

采用国际标准中不同尺寸的纸张,由于高宽比例固定为$\sqrt{2}$,除了切裁加开在工程意义上不改变高宽比例外,不同系列和开本的文件可以直接缩放影印而不会造成纸面图案有边缘裁切的问题。同系列纸张的缩放倍数更是2的幂。

但是美国纸张尺寸标准不按$\sqrt{2}$的高宽比例。 常用标准信纸的尺寸是 8.5 英寸\texttt{x}{ }11 英寸($\approx$ 215.9 毫米 x 279.4 毫米) 。法律用纸尺寸为8.5 英寸\texttt{x}{ }14 英寸($\approx$ 215.9 毫米 x 355.6 毫米)。美国是为数不多的不采用公制而采用英制的国家。尤为麻烦的是,计算机网络和软件方面,仍然时不时要同英寸打交道:1英寸=2.54厘米,或者$1\,'' = 2.54$ cm。
\newpage
\end{document} 
