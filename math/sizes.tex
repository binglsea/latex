\documentclass[main.tex]{subfiles}
\begin{document}
我们要作图,或为写文章写书排版,了解纸张的尺寸自然是有用的。国际标准的纸张尺寸可分为A、B、和C三种系列,其制定都跟数学有关。
在中国和欧洲,打印文件纸张的比较常用标准尺寸是A系列中的 A4,即,210毫米 × 297毫米 ($\approx$ 8.27 英寸 x 11.69 英寸)。
 德国科学家利希滕贝格(Georg Christoph Lichtenberg) 发现高宽比为$\sqrt{2}$
的矩形具有有趣的特点。
设一张矩形的纸宽为$w$,高为$h$,竖着看我们假设$w <h$。横着在高度一半处切裁为两个全等的更小矩形。小矩形不管是顺时针还是逆时针旋转 $90^\circ$,竖起来后设新的宽为$w'$,高为$h'$。则
$w' = h/2$ 且 $h' = w$。若使原矩形与新的小矩形相似,则两个高宽比须相等:
$$\frac{h}{w} = \frac{w}{h/2}.   $$
于是,$h^2= 2 w^2$, 或
 $$h= \sqrt{2} w.$$
这里我们又遇到了一个无理数, $\sqrt{2}$。
	
\begin{kaishu}习题.\end{kaishu} 执行如下 Python 代码,确认记忆中的$\sqrt{2}$值 $1.4142135623730951\cdots$。
\begin{spacing}{0.8}
	\begin{small}
	\begin{lstlisting}[language=Python]
print("2 的平方根 = ", 2**.5)
\end{lstlisting}
\end{small}
\end{spacing}

A 系列不同开本的纸张从一平方米(1,000,000 平方毫米)的$A_0$开始。设宽为$w_0$ 毫米,则高为
$\sqrt{2}w_0$ 毫米。于是$\sqrt{2}w_0^2 = 1000000$。
由此,
$w_0 = \sqrt{1000000/\sqrt{2}}=\sqrt{1000000}/\sqrt[4]{2}, 
h_0=\sqrt[4]{2}\cdot\sqrt{1000000}$。
几行Python代码可算出 A 系列不同开本纸张的尺寸。


\begin{spacing}{0.8}
	\begin{small}
	\begin{lstlisting}[language=Python]
#%% A系列纸张的理论尺寸
w = 1_000_000**.5 / 2**.25 #下划线"_"分隔仅为阅读方便,无实际编程作用
h = 1_000_000**.5 * 2**.25
print("A系列纸张的理论尺寸:")
for i in range(0, 11):
  # {:>3} 表示 3 个字符向右看齐。
  # {:3.0f}为浮点数的整数部分预置3个位置向右看齐,小数点后面保留0位数
  print("{:>3}: {:3.0f} 毫米 x {:4.0f} 毫米".format('A'+str(i), w, h))
  tmp = w  #暂存原宽度备作下一个高度
  w = h/2   #新宽度为原高度的一半
  h = tmp   #新高度为原宽度
\end{lstlisting}
\end{small}
\end{spacing}

A、B、和C三种系列中同系列但不同尺寸的纸张,其几何比例相同,因此可以直接缩放影印而不会造成纸面图案有边缘裁切的问题。
A系列[编辑]
A系列的制定基础首先是求取一张宽高比为
2
且面积为1平方米(m²)的纸张。因此这张纸的宽长分别为 841 毫米和 1189 毫米(宽高比为
2



186mm x 240mm
787 x 1092 1/32 


美国常用标准信纸的尺寸是 8.5 英寸 × 11 ($\approx$ 215.9 毫米 x 279.4 毫米) 。
\newpage
\end{document} 
